\documentclass{article}

\usepackage[margin=1in,a4paper]{geometry}
\usepackage[utf8]{inputenc}
\usepackage[T1]{fontenc}
\usepackage{bm}
\usepackage{lmodern}
\usepackage{tabularx}
\usepackage{fancyhdr}
\usepackage{graphicx}
\usepackage{nicefrac}
\usepackage{sectsty}
\usepackage{graphicx}
\usepackage[T1]{fontenc}
\usepackage{epigraph} %quotes
\usepackage{amsmath, amssymb} %math symbols
\usepackage{mathtools} %more math stuff
\usepackage{amsthm} %theorems, proofs and lemmas
\usepackage{optidef} %fast optimization problem notation
\usepackage{nicefrac}
\usepackage[backend=bibtex, style=numeric]{biblatex}

\usepackage[ruled,vlined,noend,linesnumbered]{algorithm2e} %algoritms/pseudocode
\let\oldnl\nl% Store \nl in \oldnl
\newcommand{\nonl}{\renewcommand{\nl}{\let\nl\oldnl}}% Remove line number for one specific line in algorithm

\usepackage{tikz}


\usepackage{hyperref}
\hypersetup{
    colorlinks,
    citecolor=black,
    filecolor=black,
    linkcolor=black,
    urlcolor=black
}

%% Some shorthands
\newcommand{\1}{\bm{1}} % vector 1 in bold
\newcommand{\bx}{\bm{x}} % vector x in bold
\newcommand{\bb}{\bm{b}} % vector b in bold
\newcommand{\bu}{\bm{u}} % vector u in bold
\newcommand{\bv}{\bm{v}} % vector v in bold
\newcommand{\bpi}{\bm{\pi}} % vector \pi in bold
\newcommand{\balpha}{\bm{\alpha}} % vector \alpha in bold
\newcommand{\blambda}{\bm{\lambda}} % vector \lambda in bold
\newcommand{\dalpha}{\dot{\balpha}} % time derviative \alpha
\newcommand{\dx}{\dot{\bx}} % time derviative x
\newcommand{\dxi}{\dot x_i} % time derviative x ith component
\newcommand{\cL}{\bar{\mathcal{L}}} % curvy L
\newcommand{\cW}{\mathcal{W}} % curvy W
\newcommand{\cI}{\mathcal{I}} % curvy I
\newcommand{\R}{\mathbb{R}} % real number
\DeclareMathOperator{\diag}{diag}


%% declaring abs so that it works nicely
\DeclarePairedDelimiter\abs{\lvert}{\rvert}%
\DeclarePairedDelimiter\norm{\lVert}{\rVert}%

% Swap the definition of \abs* and \norm*, so that \abs
% and \norm resizes the size of the brackets, and the 
% starred version does not.
\makeatletter
\let\oldabs\abs
\def\abs{\@ifstar{\oldabs}{\oldabs*}}
%
\let\oldnorm\norm
\def\norm{\@ifstar{\oldnorm}{\oldnorm*}}
\makeatother

%% Theorem notation
\newtheorem{theorem}{Theorem}[section]
\newtheorem{corollary}{Corollary}[theorem]
\newtheorem{lemma}[theorem]{Lemma}
\newtheorem{problem}{Problem}[section]
\newtheorem{definition}{Definition}[section]
\newtheorem{claim}{Claim}[section]

\pagestyle{fancy}
\fancyhf{}
\title{A predictive method for full-pose predictive distributed leader-follower formation control for non-holonomic robots}
\author{Titouan Renard, 2021}
\rfoot{Page \thepage}

\bibliography{biblio}

\begin{document}

\maketitle

\section{Problem Statement}

We have a team of $N$ differential-wheeled robots $\mathcal{R}_1, ..., \mathcal{R}_n$ described by the kinematic equations (the robot's dynamics are neglected) :
\begin{align}
    \dot{f}(\vec{x},\vec{u}) = 
    \begin{cases}
        \dot{x}_i = u_i \cos \theta_i \\
        \dot{y}_i = u_i \sin \theta_i \\
        \dot{\theta}_i = \omega_i
    \end{cases}.
\end{align}
Where $\vec{u}_i = [u_i,\omega_i]^T$ is the control input vector of $\mathcal{R}_i$ with $u_i$ linear translational speed and $\omega_i$ rotational speed. And where $\vec{x}_i = [x_i,y_i,\theta_i]^T$ is the pose vector of $\mathcal{R}_i$. We denote the full pose and control input of the system as :
\begin{align}
    \vec{x} = [x_1,y_1,\theta_1,x_2,y_2,\theta_2,...,x_N,y_N,\theta_N]^T \\
    \vec{u} = [u_1,\omega_1,u_2,\omega_2,...,u_N,\omega_N]^T
\end{align}
We denote $\textbf{R} = \{\mathcal{R}_1, ..., \mathcal{R}_n\}$ the set of all robots. Each robot $\mathcal{R}_i$ has a set of \textit{neighboring robots} $\mathcal{N}_i \subseteq \textbf{R}$, which contains the set of robots for which $\mathcal{R}_i$ can get a position estimation. The pose of $\mathcal{R}_j$ estimated by $\mathcal{R}_i$ is given by a range $\rho_{ij}$ and a bearing $\alpha_{ij}$. Each pose estimation is affected by noise which is denoted $\epsilon_z$ and is denoted by a vector  :
\begin{align}
    z_{ij} = \begin{bmatrix}
        \tilde{\rho}_{ij} \\
        \tilde{\alpha}_{ij} 
    \end{bmatrix}
    = \begin{bmatrix}
        {\rho}_{ij} \\
        {\alpha}_{ij} 
    \end{bmatrix} + \epsilon_z.
\end{align}
At time $t$ robot $\mathcal{R}_i$ gathers an observation list :
\begin{align}
    \mathcal{Z}_i = \{z_{ij} | \mathcal{R_j} \in \mathcal{N}_i\}.
\end{align}
Our goal is to have robots $\mathcal{R}_2, ... , \mathcal{R}_N$ (that we call \textit{followers}) maintain formation with the robot $\mathcal{R}_1$ (which we call \textit{leader}) while avoiding obstacles in their trajectories. We look for a control law that can be implemented in a distributed fashion for robots $\mathcal{R}_2, ... , \mathcal{R}_N$, while the control law of $\mathcal{R}_1$ is defined arbitrarily. 
\BlankLine
The formation is defined by a set of \textit{biases} $\vec{\beta}_{i} = [\delta^x_i,\delta^x_j,\delta^\theta_i]^T$, $\forall i = 2...N$ which denotes the expected pose of $\mathcal{R}_i$ relative to $\mathcal{R}_1$ within the formation. We can thus express the \textit{pose error} $\bar{x}_i$ for $\mathcal{R}_i$ as :
\begin{align}
    \bar{x}_i = \vec{x}_i - \vec{\beta}_i = \begin{bmatrix}
         x_i - \delta^x_i \\
         y_i - \delta^y_i \\
        \theta_i - \delta^\theta_i 
    \end{bmatrix}
\end{align}

The problem of maintaining formation thus becomes the problem of reducing the \textit{total pose error} $\mathcal{E} = \sum_{2...N} \norm{\vec{e_i}}$ in a distributed fashion.

\section{Laplacian-based feedback for formation control}

Let $G=(\textbf{R},\textbf{E})$ be an undirected graph constructed such that 
\begin{enumerate}
    \item it's vertex set $\textbf{R} = \{ \mathcal{R}_1, ..., \mathcal{R}_n \}$ contains every single robot in the team
    \item it's edges set contains an arbitrarily oriented edge for each robot in line of sight of another 
    \[ \textbf{E}  = \{ (\mathcal{R}_i,\mathcal{R}_j) | \mathcal{R}_j \in \mathcal{N}_i \}. \]
\end{enumerate}

Let $\mathcal{I}$ denote the \textit{incidence} matrix (with arbitrary orientations) of $G$ and $\mathcal{W}$ it's weight matrix. We compute the \textit{weighted laplacian matrix} of $G$ as follows :
\[ \mathcal{L} = \mathcal{I} \cdot \mathcal{W} \cdot \mathcal{I}^T \]
Note that the \textit{weighted laplacian matrix} $\mathcal{L}$ is constructed in such a way that :
\begin{align}
    \dot{\vec{x}} = - \mathcal{L} x(t) \\
    \dot{x_i} = \sum_{\mathcal{R}_j \in \mathcal{N}_i} w_{ij} (x_j-x_i)
\end{align}

A standard approach to formation control is to implement a Laplacian based feedback equation (which can be tough of as a PI controller) such as: 
\begin{align}
    \dot{x} = - \mathcal{L} \bar{x} + K_I \int_0^t \mathcal{L}(\tau) \bar{x}(\tau) d\tau
\end{align}

\subsection{Predictive approach to the laplacian-based formation control problem}
We propose to apply an optimization based predictive control law to our robots : 

\begin{mini}
    {u}{J(\vec{u}) = \int^{t+\tau}_{t} L(\tau,\vec{x},\vec{u}) + V(t+T,\vec{x})}{}{}
    \addConstraint{}{\dot{f}(\vec{x},\vec{u})}{}{}
\end{mini}


\section{Theoretical analysis}
In this section, we will use a \emph{normalized} Laplacian matrix $\cL \in \R^{n \times n}$ (also called the random walk normalized Laplacian). Suppose we are given the standard Laplacian matrix $\mathcal{L}$ whose diagonal elements are represented by $D = \diag(\mathcal{L}) \in\R^{n \times n}$. Then, the normalized Laplacian matrix is defined as
\[
  \cL =  D^{-1} \mathcal{L} \,.
\]
If the original Laplacian $\cL$ was unweighted, then the normalized Laplacian has the following form
\[
    \cL(i,j) = \begin{cases}
        1 \text{ if } i=j\,,\\
        -\nicefrac{1}{d_i} \text{ for } i \neq j\,. 
    \end{cases}
\]
Here, $d_i$ is the degree (number of neighbors) of the $i$th node. We summarize below some well known properties of the $\cL$ operator.

\begin{claim}[\cite{horaud2009short, wilmer2009markov}]
    The normalized Laplacian matrix satisifes
    \begin{enumerate}
        \item It is row stochastic satisfying $\sum_{j=1}^n \cL(i,j) = 1 \text{ for all } i \in [n]$.
        \item It has all real eigenvalues with the smallest being 0, of the form $(\lambda_1=0, \lambda_2, \dots, \lambda_n)$.
        \item The right eigenvector corresponding to eigenvalue 0 is the all 1s vector $\tfrac{1}{\sqrt{n}}\1$.
        \item All the eigenvalues lie in range $\lambda_1 \in [0,1]$ and $\lambda_2 > 0$ iff the graph is connected.
    \end{enumerate}
\end{claim}


\subsection{Laplacian based proportional response}
Let us consider the simple proportional response algorithm with bias which performs the update
\begin{align}\label{eqn:p-update}
 \dx = -\cL \bx + \bb\,.
\end{align}
Here, $\bb$ encodes some bias about the final formation we want our system to converge to.

\begin{claim}
    Let $\bpi$ be the left eigenvector of the matrix $\cL$ corresponding to eigenvalue 0, and let $\bpi^\top \bb = 0$. Then, assuming that $\lambda_2 > 0$ (i.e. the graph is connected) the update \eqref{eqn:p-update} coverges to the solution
    \[
        \bx(\infty) =   \alpha\1 + \cL^{\dagger} \bb \text { where } \alpha = \frac{\bpi^\top \bx(0)}{\bpi^\top \1}\,.
    \]
    Further, the formation converges to this limit at an exponential rate
    \[
        \norm{\bx(t) - \bx(\infty)}_2 \leq \exp( - \lambda_2 t)\norm{\bx(0) - \bx(\infty)}_2\,.
    \]
\end{claim}
\begin{proof}
    First note that all the eigenvalues of $-\cL$ lie in $[-1,0]$, and so the trajectory of $\bx(t)$ is stable. Now, since all the eigenvalues of $\cL$ are real, we can perform an SVD decomposition as follows
    \[
        \cL = U \Lambda V^\top  = 0 (\1 \bpi^\top) + \sum_{i=2}^n \lambda_i \bu_i \bv_i^\top \,,
    \]
    where $\Lambda$ is a diagonal matrix consisting of eigenvalues, and $U$ and $V$ are orthonormal matrices. 
    Notice that since $\bpi^\top \cL = 0$, the $\bpi$ weighted average of $\bx(t)$ remains constant over time:
    \[
      \frac{d \bpi^\top \bx}{dt} = \bpi^\top\dx = -(\bpi^\top\cL)\bx + \bpi^\top \bb = 0\,.
    \]
    This means that
    \begin{equation}\label{eqn:conserved}
        \bpi^\top\bx(t)  = \bpi^\top \bx(0)\,.  
    \end{equation}
    Next, observe that the only fixed point of the update equation \eqref{eqn:p-update} is of the form
    \[
        \cL\bx(\infty) = \bb \Rightarrow \bx(\infty) = \cL^{\dagger}\bb + \alpha \1\,,
    \]
    where $\cL^\dagger$ represents the pseudo-inverse and $\alpha$ is some constant.
    Using the previous constraint on the weighted sum of $\bx(t)$, we can compute the value of $\alpha$ as
    \[
            \bpi^\top \cL^{\dagger}\bb + \alpha \bpi^\top\1  = \alpha \bpi^\top\1 \Rightarrow \alpha = \frac{\bpi^\top \bx(0)}{\bpi^\top \1}\,.
    \]
    This proves the first part of the claim. We can now solve the matrix differential equation above to write 
    \[
      \bx(t) - \bx(\infty) =   \exp(- \cL t) (\bx(0) - \bx(\infty)) = (\exp(- \cL t) - \1 \pi^\top) (\bx(0) - \bx(\infty))\,.
    \]
    The last equality follows from our observation \eqref{eqn:conserved}. Now, taking euclidean norms and both sides gives
    \[
        \norm{\bx(t) - \bx(\infty) }_2 \leq  \norm{\exp(- \cL t) - \1 \pi^\top}_{2}\norm{\bx(0) - \bx(\infty)}_2 = \exp(-\lambda_2 t)\norm{\bx(0) - \bx(\infty)}_2\,.
    \]
    This finishes our proof.
    % since $\bpi^\top \bb = 0$, we have the following for $\cL^{\dagger}$ which is the pseudo-inverse of $\cL$:
    % \[
    %     \cL \cL^{\dagger} \bpi = \bpi\,.
    % \]  
    % Thus the update equation can be simplified as 
    % \[
    %   \dx = -\cL( \bx - \cL^{\dagger}\bb )\,.  
    % \]

    % Next, let us rewrite $(\bx(t) - \cL^{\dagger}\bb)$ as $\balpha(t)$ in the coordinate system of $V$:
    % \[
    %   \bx(t) = \frac{\alpha_1(t)}{\sqrt{n}}\1 + \sum_{i=2}^n \alpha_i(t) \bv_i  + \cL^{\dagger}\bb \,.
    % \]
    % Then, we can simplify as
    % \[
    %     \dx = -\cL (\bx(t) - \cL^{\dagger}\bb) =  -(\sum_{i=2}^n \lambda_i \bu_i \bv_i^\top)(\frac{\alpha_1(t)}{\sqrt{n}}\1 + \sum_{i=2}^n \alpha_i(t) \bv_i) = -\sum_{i=2}^n \lambda_i \alpha_i(t) \bu_i\,,
    % \]
    % Recall that $\bx(t) = V \balpha(t) + \cL^{\dagger}\bb$ and so we can rewrite the above equation as
    % \[
    %     \dx = - U \Lambda \balpha(t) \Rightarrow \dx = - U \Lambda V^\top (\bx(t) - \cL^{\dagger}\bb)\,.
    % \]
    % Solving this matrix differential equation yields 
    % \[
    %   \bx(t) =   
    % \]
\end{proof}

\printbibliography

\end{document}  
